\PassOptionsToPackage{unicode=true}{hyperref} % options for packages loaded elsewhere
\PassOptionsToPackage{hyphens}{url}
%
\documentclass[
]{article}
\usepackage{lmodern}
\usepackage{amssymb,amsmath}
\usepackage{ifxetex,ifluatex}
\ifnum 0\ifxetex 1\fi\ifluatex 1\fi=0 % if pdftex
  \usepackage[T1]{fontenc}
  \usepackage[utf8]{inputenc}
  \usepackage{textcomp} % provides euro and other symbols
\else % if luatex or xelatex
  \usepackage{unicode-math}
  \defaultfontfeatures{Scale=MatchLowercase}
  \defaultfontfeatures[\rmfamily]{Ligatures=TeX,Scale=1}
\fi
% use upquote if available, for straight quotes in verbatim environments
\IfFileExists{upquote.sty}{\usepackage{upquote}}{}
\IfFileExists{microtype.sty}{% use microtype if available
  \usepackage[]{microtype}
  \UseMicrotypeSet[protrusion]{basicmath} % disable protrusion for tt fonts
}{}
\makeatletter
\@ifundefined{KOMAClassName}{% if non-KOMA class
  \IfFileExists{parskip.sty}{%
    \usepackage{parskip}
  }{% else
    \setlength{\parindent}{0pt}
    \setlength{\parskip}{6pt plus 2pt minus 1pt}}
}{% if KOMA class
  \KOMAoptions{parskip=half}}
\makeatother
\usepackage{xcolor}
\IfFileExists{xurl.sty}{\usepackage{xurl}}{} % add URL line breaks if available
\IfFileExists{bookmark.sty}{\usepackage{bookmark}}{\usepackage{hyperref}}
\hypersetup{
  pdfborder={0 0 0},
  breaklinks=true}
\urlstyle{same}  % don't use monospace font for urls
\usepackage[margin=1in]{geometry}
\usepackage{graphicx,grffile}
\makeatletter
\def\maxwidth{\ifdim\Gin@nat@width>\linewidth\linewidth\else\Gin@nat@width\fi}
\def\maxheight{\ifdim\Gin@nat@height>\textheight\textheight\else\Gin@nat@height\fi}
\makeatother
% Scale images if necessary, so that they will not overflow the page
% margins by default, and it is still possible to overwrite the defaults
% using explicit options in \includegraphics[width, height, ...]{}
\setkeys{Gin}{width=\maxwidth,height=\maxheight,keepaspectratio}
\setlength{\emergencystretch}{3em}  % prevent overfull lines
\providecommand{\tightlist}{%
  \setlength{\itemsep}{0pt}\setlength{\parskip}{0pt}}
\setcounter{secnumdepth}{-2}
% Redefines (sub)paragraphs to behave more like sections
\ifx\paragraph\undefined\else
  \let\oldparagraph\paragraph
  \renewcommand{\paragraph}[1]{\oldparagraph{#1}\mbox{}}
\fi
\ifx\subparagraph\undefined\else
  \let\oldsubparagraph\subparagraph
  \renewcommand{\subparagraph}[1]{\oldsubparagraph{#1}\mbox{}}
\fi

% set default figure placement to htbp
\makeatletter
\def\fps@figure{htbp}
\makeatother


\author{}
\date{\vspace{-2.5em}}

\begin{document}

\hypertarget{materials-and-methods}{%
\section{Materials and methods}\label{materials-and-methods}}

\hypertarget{general}{%
\subsection{General}\label{general}}

\hypertarget{ethic-approval}{%
\subsubsection{Ethic approval}\label{ethic-approval}}

All procedures were conducted in accordance with the directive
2010/63/EU of the European Parliament and of the Council of 22 September
2010 on the protection of animals used for scientific purposes.
Procedures in immunostaining and patch-clamp experiments were
additionally conducted in accordance with the Polish Act on the
Protection of Animals Used for Scientific or Educational Purposes of 15
January 2015, and approved by the 2nd Local Ethics Commissions on Animal
Research (Krakow, Poland). All efforts were made to minimize suffering
and to reduce the number of animals used.

\hypertarget{animals}{%
\subsubsection{Animals}\label{animals}}

Male Wistar rats (50-180g on arrival) were purchased from Charles River
Laboratories (Research Models and Services; Germany). Four to five
animals were housed per polypropylene cage (55x35x20 cm) in controlled
environment (22 ± 1\(^{\circ}\)C, 45 ± 5 \% relative humidity, 12:12 h
light/dark cycle, lights on at 7:00 a.m.) with commercial food and fresh
water \emph{ad libitum}. Animals were acclimated to the housing
environment for at least 4 days before the beginning of the experiments.
Rats were weighing 180-300g during the tissue preparation / 5-10 weeks
of age. Animals at this age are considered as young adults.

\hypertarget{reagents}{%
\subsubsection{Reagents}\label{reagents}}

\hypertarget{whole-cell-patch-clamp-technique-in-acute-brain-slices}{%
\subsection{Whole-cell patch-clamp technique in acute brain
slices}\label{whole-cell-patch-clamp-technique-in-acute-brain-slices}}

\hypertarget{tissue-preparation}{%
\subsubsection{Tissue Preparation}\label{tissue-preparation}}

Male Wistar (4-7-week-old) rats were anesthetised with isoflurane
(AErrane, Baxter, Poland) and decapitated between 07:00 and 09:00 a.m.
Brains were collected in ice-cold, low-sodium, high-magnesium ACSF
(artifical cerebrospinal--fluid), containing (in mM): 65 sucrose, 76
NaCl, 25 NaHCO\textsubscript{3}, 1.4
NaH\textsubscript{2}PO\textsubscript{4}, 25 glucose, 2.5 KCl, 7
MgCl\textsubscript{2}, 0.4 Na-ascorbate, and 2 Na-pyruvate (bubbled with
95\% O2/5\% CO2), pH 7.4; osmolality 290--300 mOsmol
kg\textsuperscript{−1}) and cut into 300 \mu m thick coronal sections on
a Leica VT 1000 vibrating microtome (Leica Instruments, Germany).
Sections containing the amygdala were transferred to an incubation
chamber containing carbogenated, warm (32\(^{\circ}\)C) ACSF, containing
(in mM): 92 NaCl, 30 NaHCO\textsubscript{3}, 2.5 KCl, 1.2
NaH\textsubscript{2}PO\textsubscript{4}, 2 CaCl\textsubscript{2}, 20
HEPES, 2 MgSO\textsubscript{4} and 25 glucose, 5 Na-ascorbate, 2
thiourea, 3 Na-pyruvate, pH = 7.35; osmolality 290--300 mOsmol
kg\textsuperscript{−1}). After a recovery period (60-90 minutes) slices
were transferred to a recording chamber placed on a fixed stage of an
Zeiss Axioskop 2 (Zeiss, Germany) upright microscope, where the tissue
was perfused (1-2 ml/min) with carbogenated, warm (32°C) ACSF containing
(in mM): 124 NaCl, 30 NaHCO\textsubscript{3}, 2.5 KCl, 1.2
NaH\textsubscript{2}PO\textsubscript{4}, 2 CaCl\textsubscript{2}, 5
HEPES, 2 MgSO\textsubscript{4} and 10 glucose, pH = 7.35; osmolality
290--300 mOsmol kg\textsuperscript{−1}).

\hypertarget{whole-cell-patch-clamp-recordings-and-data-acquisition}{%
\subsubsection{Whole-cell patch-clamp recordings and data
acquisition}\label{whole-cell-patch-clamp-recordings-and-data-acquisition}}

Recording micropipettes were fabricated from borosilicate glass
capillaries (3--6 MΩ; Sutter Instruments, USA) using horizontal puller
(Sutter Instruments) and filled with the following solutions. In
experiments on Wistar rats the solution contained (in mM): 125 potassium
gluconate, 20 KCl, 2 MgCl\textsubscript{2}, 4 Na\textsubscript{2}ATP,
0.4 Na\textsubscript{3}GTP, 5 EGTA, 10 HEPES, pH 7.3, osmolality 290-300
mOsmol kg\textsuperscript{−1}) and biocytin (0.1\%, for subsequent
immunofluorescent identification of recorded neurons). The calculated
liquid junction potential using this solution was 12 mV, and this value
was subtracted from the data.

\hypertarget{post-recording-immunostaining}{%
\subsubsection{Post-recording
immunostaining}\label{post-recording-immunostaining}}

\hypertarget{electrophysiological-identification-and-classification-of-chosen-neurons}{%
\subsubsection{Electrophysiological identification and classification of
chosen
neurons}\label{electrophysiological-identification-and-classification-of-chosen-neurons}}

\hypertarget{electrophysiological-identification-of-principal-cells-in-the-bla}{%
\paragraph{Electrophysiological identification of principal cells in the
BLA}\label{electrophysiological-identification-of-principal-cells-in-the-bla}}

\hypertarget{electrophysiological-identification-of-late-firing-cells-in-the-cea}{%
\paragraph{Electrophysiological identification of late-firing cells in
the
CeA}\label{electrophysiological-identification-of-late-firing-cells-in-the-cea}}

\hypertarget{electrophysiological-identification-of-regular-firing-cells-in-the-cea}{%
\paragraph{Electrophysiological identification of regular-firing cells
in the
CeA}\label{electrophysiological-identification-of-regular-firing-cells-in-the-cea}}

\end{document}
